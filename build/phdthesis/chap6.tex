\chapter{결론}

%% A generation of dynamic languages have been designed by trying variants of
%% the class-based object oriented paradigm. This process has been aided by
%% the development of standard techniques (e.g. bytecode VMs) and reusable
%% infrastructure such as code generators, garbage collectors, and whole VMs
%% like the JVM and CLR.

%% It is possible to envision a future generation of languages that generalize
%% this design to set-theoretic subtyping instead of just classes. This next
%% generation will require its own new tools, such as partial evaluators (already
%% under development in PyPy and Truffle). One can also imagine these future
%% language designers wanting reusable program analyses, and tools for
%% developing lattices and their operators.

테크니컬 컴퓨팅이 무엇이다고 정확히 정의하기는 어려운 일이다.
그런 이유로, 우리는 목표를 ``특정 영역''에 한정되게 잡지 않았다.
프로그래밍 언어는 어떠한 목적으로도 사용할 수 있는 보편성이 중요하다고
프로그래밍 언어 도입\cite{Meyerovich:2013:EAP:2509136.2509515} 을 주제로 한 최근의 논문은 말한다.
특수한 분야에서만 쓰이는 언어는 보통 그 분야에 제일 잘나가는 언어가 아니다.
여기서도 역시 보편적인 언어가 선호되며 너른 인기를 끌고 있다.
시장을 지배할 만큼 잘 만드는게 결코 쉽지 않다는 것은, 우리가 만드는
보편적인 언어에서 이를 개선할 여지가 많이 있음을 반증한다.
서두에서 언급한 것 처럼, 테크니컬 컴퓨팅의 모습은 부정확하게 정의되었고, 그저 되는 대로 만들어졌을 뿐이다.
이제 보다 보편적인 틀에서 이 분야를 이해하기 위한 한걸음을 내딛어 보자.

%we see it as a motivation, not a target
%want to be more general, not domain specific

%``I've yet to hear anyone explain how you decide what are the boundaries of a `domain-specific' language. Isn't the `domain' mathematics and science itself?''~\cite{bobharperdsl}

%tc is not a domain, and doesnt need to be:
% \cite{meyerovich2012socio} about appealing to some domain first
% technical computing was ripe for such a treatment

테크니컬 컴퓨팅 사용자들은 우선 처리할 일을 프로그래밍 언어로서 직접 기술하는게
드물다. 우리는 이런 상황을 지켜보며 이 프로젝트를 시작하였다.
현재 많은 수의 테크니컬 컴퓨팅 언어가 이런 기능을 나열하고 있긴 하지만 말이다.
% in particular, they want flexible notation, and code specialized
% for information regardless of when that information is known.
어떻게 보면 부분적인 정보(partial information)를 표기할 적절한 방법이 없는 문제다.
부분적인 정보는 여러 상황에서 쓸 수 있다.
각종 값을 설명하고 코드에 적용할 때나
컴파일러가 결정할 성능 모델에 영향을 줄 각종 정보를 다룰 때.
스테이지 프로그래밍을 이끌 때: 코드 생성은 전체를 대상으로 하는게 아닌, 계산을 필요로 하는 지점만을 파악한다,
그래서 생성된 코드는 능히 재사용 할 수 있게 된다.
이는 결국 타입 시스템이 하는 일이다.
그렇지만 우리가 관심을 갖는 영역의 프로그래머, 동적 언어를 쓰는 사람이
강하게 거부하는 것이기도 하다.
보다 많이 타입을 사용하도록 만들 수 있으며 그러한 변화가 곳곳에서 발견되고 지지되고 있다.
물론 부분적인 정보를 기반으로 런타임에 유용한 행위를 얻는 다른 많은 방법이 있다.

이 장에서 다루지 못한 것은 더 진행을 하면서 반영할 것이다.

% what's wrong with dynamic languages is not just a lack of static
% guarantees, but that they don't seem to be pareto optimal.
% actually not enough is done with dynamic typing: you give up type
% checking, and all you get is untyped dictionaries in return?

% we know these programs need more organization and type info.
% this is the first truly viable approach
%   - bad that they need to think about type stability, but
%     also good that is relatively easy to grasp, and it gets
%     people thinking about it who wouldn't normally
%     - the types give people the tools to deal with it
% - can't pretend that people wont have to worry about this stuff;
%   it's fundamental

%exploiting partial information is key.
%dispatch was an especially natural way to do this,
%are there others?
%What else can we do with partial information as part of a programming
%model?
% somehow being dual to ML-family languages; what's hard for them is
% easy for us and vice versa.

\section{성능}

추상화에 관심이 있다면, 성능을 위해서 우선 타입 최적화(type specialization)를 해야 한다.
하지만 크게 보면 성능을 끌어내기 위해서는 더욱 많은 것을 다뤄야 한다.
\emph{알고리즘 선별}~\cite{Ansel:2009:PLC:1542476.1542481,Ansel:2014:OEF:2628071.2628092} 아이디어에서
다루는 ``코드 선택(code selection)''을 보면, 예를 들어 프로그램의 어떤 지점에 몇몇의 알고리즘을 선별해서 가장 빠른 것을 자동으로 고른다.
줄리아는 작은 단위로 간결하게 함수를 작성하고, 적용하고 있는데
자연스레 이러한 기능과 맞아 떨어진다.
객체의 실행 단위별로 디스패치 하는 것도 시도해 볼 만하다.

Spiral~\cite{Puschel:2004:SGP:1080647.1080687} 과 같은 고성능 코드 생성기는
데이터의 크기를 최적화 한다.
줄리아는 배열 크기를 표현하기 쉽도록 타입 시스템이 고안되어 있으므로, 코드를 최적화하는
동적 디스패치 패턴으로서 이를 적용해 볼 수 있다.
% immutable arrays, specializing on size

%There are several key aspects of performance programming that our design
%does not directly address, e.g. storage and in-place optimizations.

줄리아는 분산 컴퓨팅 라이브러리를 포함하고 있으며, 메모리 공유 방식의 병렬화에 대한
탐색은 아직 충분히 하지 못하였다. Cilk 모델~\cite{Joerg96,Randall98} 은 우리가 목표로 하는
성능과 추상화에 대한 좋은 본보기다.


\section{앞으로 할 작업들}

실무용 시스템에서는, 초기 디자인에서 버려둔 것을 다시 돌아보는 작업을 해야 한다.
최적화를 기반으로 하는 다변성(polymorphism)에서 우리는 분할 컴파일(separate compilation) 기능을 지원하지 않고 있다.
다행히 그런 시도를 멈춘 적은 없다: C++ 템플릿은 같은 문제를 갖고 있는데,
다만 컴파일된 모듈을 개별적으로 생성할 수는 있다.
우리는 같은 방식으로 해 낼 계획인데, 각 모듈의 관련 코드를 다시-분석하는 수고만으로 말이다.
또 다른 접근법은 모듈을 분리하는 대신 단계(layer)를 사용하는 것이다.
``봉인한(sealed)'' 단계라 더 이상 기능 확장이 안되게 ---
이는 언어의 한 기능으로도 적합하다.
이 라이브러리를 사용하는 애플리케이션은 보다 효율적으로 컴파일 할 수 있다.
단계를 추가하여(느린 것부터 바꾸는) 특정 상황에 적절한 지원을 한다. 
``텔레스코핑 언어''는 이것을 핵심~\cite{telescoping} 으로 다룬다.

줄리아에서 고차원적 프로그래밍을 하는 이상적인 접근은 난제로 남아 있다.
~\ref{sec:implementingmap} 의 \texttt{map} 함수는 \texttt{Bottom} 타입을
원소로 하는 배열을 돌려준다. 입력이 비어있는 경우를 대처할 수 있지만 여전히
많은 프로그래머들을 헤깔리게 하고 당황하게 만든다.
애로우 타입(nominal arrow types)을 위한 별도의 문법을 지원하여
이를 처리할 수 있을 것이다.

정적 타입 점검 또한 생각해 볼 문제다.
이건 굉장히 중요한 문제라 중요도 순으로: 첫째는 타입 정보를 나타내도록 하고,
점진적으로 점검해 나가며 안전성을 향상해 간다.
예컨데, 언어적으로 더 나은 대안을 찾으려 한다면 제어 흐름(control flow)과
맞물리는 타입을 찾아낼 필요가 있다.
\texttt{TypeCheck.jl}~\cite{typecheckjl} 와 같은 도구는 추론할 수 있는
타입 정보로 무엇이든 이를 활용해 쓸 수 있다.

%There are many opportunities to incorporate more detailed static analysis.

% higher resolution type info degrades SNR

%% \begin{itemize}
%% \item formalize more details of dispatch, meet, join, widen
%% \item product domains, how to incorporate finer types more smoothly
%% *\item how to incorporate static checks? \cite{typecheckjl}
%% *\item separate compilation
%% *\item function types
%% \item shared memory, cilk
%% \item can we implement BLAS?
%% *\item incorporate autotuning?
%% \end{itemize}

% oscar:
% - what we have now is the ``minimal julia''
% - ``easy to write code that's easy to analyze statically''
% - got non-programmer people to write easy to analyze code

% mention lines of code
%% How much of the past 30 years of handwritten Matlab internals can
%% be autogenerated with a compiler? (A lot)
%% 70kLOC julia, 34kLOC c/c++, 6kLOC scheme
% but this includes a REPL, package manager

% easy polymorphism.
% - starts with a very simple HLL
% - performs well because of underlying mechanisms,
%   which then unfold gradually when you want more control over performance

% making everything a method, and everything glued together by types
% gives certain reflective properties.


\section{프로젝트 상황}

줄리아는 자유 소프트웨어이며, 우리는 대부분의 코드를 작성하고 읽고 고치는 과정을
진두지휘 하고 있다. 줄리아의 대부분은 라이브러리로서 존재하며, 시스템을 아울러
누구나 기여를 할 수 있게 진입 장벽을 낮추었다.
현재 590개의 패키지가 패키지 관리 시스템에 등록되어 있고, 대부분 줄리아로 쓰여졌다.


%over 350 contributors


% something about how ``information hiding'' is anathema to science,
% and coincidentally is not the focus of t.c. languages or rigorous
% functional languages.

